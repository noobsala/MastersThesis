% T I T L E   P A G E
% -------------------
% Last updated May 24, 2011, by Stephen Carr, IST-Client Services
% The title page is counted as page `i' but we need to suppress the
% page number.  We also don't want any headers or footers.
\pagestyle{empty}
\pagenumbering{roman}

% The contents of the title page are specified in the "titlepage"
% environment.
\begin{titlepage}
        \begin{center}
        \vspace*{1.0cm}

        \Huge
        {\bf Up-Scaling Discrete Element Method Simulations of Discontinua}

        \vspace*{1.0cm}

        \normalsize
        by \\

        \vspace*{1.0cm}

        \Large
        Mike Yetisir \\

        \vspace*{3.0cm}

        \normalsize
        A thesis \\
        presented to the University of Waterloo \\ 
        in fulfillment of the \\
        thesis requirement for the degree of \\
        Master of Applied Science \\
        in \\
        Civil Engineering \\

        \vspace*{2.0cm}

        Waterloo, Ontario, Canada, 2016 \\

        \vspace*{1.0cm}

        \copyright\ Mike Yetisir 2016 \\
        \end{center}
\end{titlepage}

% The rest of the front pages should contain no headers and be numbered using Roman numerals starting with `ii'
\pagestyle{plain}
\setcounter{page}{2}

\cleardoublepage % Ends the current page and causes all figures and tables that have so far appeared in the input to be printed.
% In a two-sided printing style, it also makes the next page a right-hand (odd-numbered) page, producing a blank page if necessary.
 


% D E C L A R A T I O N   P A G E
% -------------------------------
  % The following is the sample Delaration Page as provided by the GSO
  % December 13th, 2006.  It is designed for an electronic thesis.
  \noindent
I hereby declare that I am the sole author of this thesis. This is a true copy of the thesis, including any required final revisions, as accepted by my examiners.

  \bigskip
  
  \noindent
I understand that my thesis may be made electronically available to the public.

\cleardoublepage
%\newpage

% A B S T R A C T
% ---------------

\begin{center}\textbf{Abstract}\end{center}

Pre-existing fractures significantly influence the geomechanical response of the rock mass at the reservoir scale. For geomechanical applications, these natural fractures need to be considered in the mechanical response of the system. Distinct Element Methods (DEM) are often used to explicitly model the mechanics of Naturally Fractured Rock (NFR); however, they are often too computationally prohibitive for reservoir-scale problems. A DEM up-scaling framework is presented that facilitates estimating a representative parameter set for continuum constitutive models that capture the salient feature of Naturally Fractured Rock (NFR) behaviour. 

Up-scaling is achieved by matching homogenized DEM stress-strain curves from multiple load paths to those of continuum constitutive models using a Particle Swarm Optimization (PSO) algorithm followed by a Damped Least-Squares (DLS) algorithm. The effectiveness of the framework is demonstrated by up-scaling a DEM model of a NFR to a Drucker-Prager damage-plasticity model; the up-scaled model is shown to capture well the effect of confinement on the the yielding and sliding of natural fractures in the rock mass. 

The goal of this thesis is to present a framework to facilitate effective simulation of fine-scale behaviour in full-scale NFR systems while significantly reducing the computational demands associated with modelling these systems with DEM. 

As such, four main research objectives have been identified and achieved: 1) Develop and implement stress and strain homogenization algorithms for DEM models with deformable blocks, 2) present a methodology to parameterize complex nonlinear continuum constitutive models, 3) develop and implement an automated modular software framework for up-scaling DEM simulations, and 4) demonstrate that the performance of the up-scaled continuum models are accurate and significantly more computationally efficient.

The up-scaling methodology is verified through a case study on a naturally fractured granite slope in which the top surface is loaded until failure. The up-scaled continuum model is shown to compare quite well to Direct Numerical Simulation (DNS) in a slope stability analysis and requires two orders of magnitude less computational effort.

\cleardoublepage
%\newpage

% A C K N O W L E D G E M E N T S
% -------------------------------

\begin{center}\textbf{Acknowledgements}\end{center}

This thesis, though perhaps at times a lonely affair, could not have been possible without the involvement and support of many individuals who helped and encouraged me at every step along the way. I wish to acknowledge the invaluable support and contributions that everyone provided. 

First and foremost, I wish to thank my two research supervisors, Dr. Rob Gracie and Dr. Maurice Dusseault, for their endless technical insights, inspirational ideas, and moral support throughout the duration of this degree. With out their guidance, this research thesis would not have been nearly as comprehensive nor comprehensible.

Additionally, I would like to acknowledge Dr. L. Shawn Matott for allowing me to use his OSTRICH optimization software and providing some much needed support when I was getting started. 

Furthermore, I should thank my partners in crime (and research), Endrina Rivas and Eleanor Mak, with whom I have shared many late nights at the office. These ladies have kept me motivated when my research has stagnated, while conversely making sure I never worked too hard. 

Lastly, additional thanks goes to my coaches, Vinit Kudva, Clive Porter, and Jeff Muirhead for pushing me to be my best on and off the court, distracting me from my studies, and allowing me to be part of such a wonderfull team.

\cleardoublepage
%\newpage

% % D E D I C A T I O N
% % -------------------

% \begin{center}\textbf{Dedication}\end{center}

% This thesis is dedicated to

% \cleardoublepage
% %\newpage

% T A B L E   O F   C O N T E N T S
% ---------------------------------
\renewcommand\contentsname{Table of Contents}
\tableofcontents
\cleardoublepage
\phantomsection
%\newpage

% L I S T   O F   T A B L E S
% ---------------------------
\addcontentsline{toc}{chapter}{List of Tables}
\listoftables
\cleardoublepage
\phantomsection		% allows hyperref to link to the correct page
%\newpage

% L I S T   O F   F I G U R E S
% -----------------------------
\addcontentsline{toc}{chapter}{List of Figures}
\listoffigures
\cleardoublepage
\phantomsection		% allows hyperref to link to the correct page
%\newpage

% L I S T   O F   S Y M B O L S
% -----------------------------
%To include a Nomenclature section
\addcontentsline{toc}{chapter}{List of Abbreviations}
\printglossary[title={List of Abbreviations}, style=list, nogroupskip]
\cleardoublepage
\phantomsection % allows hyperref to link to the correct page
%\newpage

% Change page numbering back to Arabic numerals
\pagenumbering{arabic}

